\documentclass[a4paper]{article}

\usepackage[utf8]{inputenc}

\usepackage[francais]{babel}

\newcommand{\vtitle}[0]{Représentation et analyse du réseau de confiance OpenPGP}

\usepackage{anysize}

\marginsize{2cm}{2cm}{1cm}{2cm}


\usepackage{url, array, listings, graphicx, framed, relsize}

\sloppy


\title{\vtitle}

\author{Antoine Amarilli}

\date{}


\usepackage{fancyhdr}

\fancypagestyle{plain}{

\fancyhf{}

\fancyhead[L]{\vtitle}

\fancyfoot[L]{Antoine Amarilli}

\fancyfoot[R]{\thepage}

\renewcommand{\headrulewidth}{0.4pt}

\renewcommand{\footrulewidth}{0.4pt}

}

\pagestyle{plain}


\newcommand{\deft}[1]{\textbf{#1}}

\newcommand{\theo}[1]{\emph{#1}}

\newcommand{\code}[2]{
\pagebreak
\begin{framed} 
\lstset{language=C, caption=#1 : #2, label=#1, numbers=left, breakatwhitespace=true, showstringspaces=false, breaklines=true, basicstyle=\scriptsize, }
\lstinputlisting{#1}
\end{framed}
}

\newcommand{\sch}[3]{
\begin{figure}
\centering
\includegraphics{#1_#2.pdf}
\caption{#3}
\label{sch_#2}
\end{figure}
}

\newcommand{\img}[3]{
\begin{figure}
\centering
\includegraphics[angle=90,scale=0.34]{#1_#2.pdf}
\caption{#3}
\label{img_#2}
\end{figure}
}

\newcommand{\tbl}[2]{
\begin{table}[p]
\centering
\input{#1.tex}
\caption{#2}
\label{#1}
\end{table}
}


\urldef\byuedu\url{byu.edu}
\urldef\unipotsdamde\url{uni-potsdam.de}

\begin{document}

\sch{schemas/sch}{alice_bob}{Schéma de principe de la communication à l'aide de la cryptographie asymétrique. Les clés A1 et A2 sont respectivement les clés privée et publique d'Alice, B1 et B2 celles de Bob. Les flèches en pointillés indiquent le transfert de données sur un canal vulnérable aux attaques passives. Le protocole assure à Alice que son message n'est lisible que par Bob, et assure à Bob l'authenticité du message reçu.}

\sch{schemas/sch}{alice_bob_mallory}{Schéma de principe de l'attaque de l'homme du milieu, menée par Mallory. Les clés A1 et A2 sont respectivement les clés privée et publique d'Alice, B1 et B2 celles de Bob, et A'1, A'2, B'1, B'2 des clés factices créées par Mallory. Les flèches en pointillés indiquent le transfert de données sur un canal vulnérable aux attaques actives. En se faisant passer pour Bob auprès d'Alice et pour Alice auprès de Bob, Mallory est en mesure de lire et de modifier le message.}

\tbl{all_tlds_to_all_tlds.custom}{Distance moyenne entre les clés d'un TLD et celles d'un autre TLD. Aucune tendance notable ne semble pouvoir être observée. Noter que le tableau n'est pas symétrique, car le graphe du réseau de confiance est orienté. Les distances sont indiquées en partant du TLD de la ligne pour aller jusqu'au TLD de la colonne.}

\sch{schemas/sch}{alice_carol_bob}{Schéma de principe du réseau de confiance. Les flèches bleues, vertes et rouges indiquent la signature de clé, la confiance en une personne, et l'assurance de la validité de la clé respectivement. Alice a vérifié la clé de Carol (elle a donc signé la clé de Carol) et a confiance en Carol, et Carol a vérifié la clé de Bob (elle a donc signé la clé de Bob), donc Alice a une garantie de la validité de la clé de Bob.}

\sch{schemas/sch}{alice_carol_dave_zoe}{Schéma de principe de la non-transitivité du réseau de confiance. La légende est celle de la figure \ref{sch_alice_carol_bob}. Alice a vérifié la clé de Carol et a confiance en elle, et celle-ci a vérifié la clé de Dave. Alice a donc une garantie de la validité de la clé de Dave, mais pas de celle de Zoé puisqu'Alice n'a pas confiance en Dave.}


\tbl{all_tlds_vs_rand}{Distance moyenne entre tout couple de clés pour chaque TLD, comparé aux distances pour un sous-ensemble aléatoire de clés de même taille. Les colonnes indiquent respectivement le nombre de clés dans le TLD, la distance moyenne entre tout couple de clés du TLD, la distance moyenne entre tout couple de clés du sous-ensemble aléatoire, et la différence de ces deux colonnes. Pour les TLD correspondant à un pays, la distance moyenne du TLD est en général plus basse que celle de l'ensemble de clés aléatoires.}


\tbl{dist_vs_gdist_before}{Distance moyenne graphique (euclidienne) entre tout couple de clés pour chaque TLD, comparé aux distances pour un sous-ensemble aléatoire de clés de même taille, avant lancement de l'algorithme force-directed. Les colonnes sont les mêmes que celles du tableau~\ref{all_tlds_vs_rand}, à ceci près qu'il s'agit ici de distances graphiques. Aucune tendance notable ne semble pouvoir être observée.}

\tbl{dist_vs_gdist_after}{Distance moyenne graphique (euclidienne) entre tout couple de clés pour chaque TLD, comparé aux distances pour un sous-ensemble aléatoire de clés de même taille, après exécution de l'algorithme force-directed pendant quelques heures. Les colonnes sont les mêmes que celles du tableau~\ref{dist_vs_gdist_before}. Pour les TLD correspondant à un pays, la distance moyenne du TLD est en général plus basse que celle de l'ensemble de clés aléatoires.}

\begin{table}[p]
\centering
\begin{tabular}{|c|l|}
\hline {\bf Entrée} & {\bf Effet} \\
\hline Clic gauche & Sélection de clé(s) \\
\hline Clic droit & Déplacement de clé(s) \\
\hline Clic central & Déplacement de la vue \\
\hline Molette & Zoom \\
\hline a & Tout sélectionner \\
\hline c & Colorier les clés \\
\hline d & Calculer les distances entre clés \\
\hline e & Sélection par TLD ou adresse de courriel \\
\hline f & Marquage des clés de départ pour les calculs de distance \\
\hline g & Affichage de la résultante des forces \\
\hline i & Affichage des identifiants de clés \\
\hline k & Sélection par identifiant \\
\hline l & Recalcul manuel de la résultante \\
\hline m & Déplacement manuel suivant la résultante \\
\hline n & Affichage des noms et adresses de courriel \\
\hline q & Quitter \\
\hline r & Sélection de clés aléatoires \\
\hline s & Sélection des clés ayant signé les clés sélectionnées \\
\hline t & Affichage du nombre de clés dans la sélection \\
\hline v & Inversion de la sélection \\
\hline x & Activation ou désactivation de l'algorithme force-directed \\
\hline z & Zoom automatique \\
\hline / & Remise à zéro des opérateurs \\
\hline + & Opérateur union \\
\hline - & Opérateur différence ensembliste \\
\hline * & Opérateur intersection \\
\hline \verb?\? & Opérateur différence symétrique \\
\hline A & Tout sélectionner \\
\hline C & Coloriage rapide \\
\hline D & Suppression des marques de départ et d'arrivée \\
\hline F & Marquage des clés d'arrivée pour les calculs de distance \\
\hline G & Masquage de la résultante des forces \\
\hline I & Masquage des identifiants de clés \\
\hline L & Recalcul manuel de la résultante (toutes les clés) \\
\hline M & Déplacement manuel suivant la résultante (toutes les clés) \\
\hline N & Masquage des noms et adresses de courriel \\
\hline S & Sélection des clés ayant été signées par les clés sélectionnées \\
\hline Z & Centrer la vue \\
\hline Ctrl+A & Calcul de données pour les tableaux \ref{all_tlds_vs_rand}, \ref{dist_vs_gdist_before} et \ref{dist_vs_gdist_after} \\
\hline Ctrl+B & Calcul de données pour le tableau \ref{all_tlds_to_all_tlds.custom} \\
\hline Ctrl+C & Arrangement des clés sélectionnées en cercle \\
\hline
\end{tabular}
\caption{Liste des commandes du logiciel.}
\label{commands}
\end{table}


\img{images/r2002}{at}{Position des clés autrichiennes (en noir) dans le réseau de confiance.}

\img{images/r2002}{rnd_at}{Position d'un sous-ensemble de clés aléatoires aussi nombreuses que les clés autrichiennes, à comparer avec la figure \ref{img_at}. On remarque que les clés autrichiennes ont davantage tendance à former des alignements.}

% \img{images/r1822}{ch}{Position des clés suisses (en noir) dans le réseau de confiance. On remarque des alignements de clés.}

% \img{images/r1822}{ch_zoom}{Zoom sur un alignement de la figure \ref{img_ch}.}

% \img{images/r1822}{keys_colors}{Ensemble du réseau de confiance, avec coloriage des clés selon leur TLD. La légende associant couleurs et TLD est indiquée par les clés dont l'adresse de courriel est affichée.}

% \img{images/r1822}{center_colors}{Zoom sur le centre du réseau de confiance (voir figure \ref{img_keys_colors}), avec coloriage des clés selon leur TLD. La répartition des TLD n'est pas la même en tous points.}

\img{images/r2014}{all_colors}{Centre du réseau de confiance, avec coloriage des clés selon leur TLD. On observe que les TLD ne sont pas répartis de façon homogène.}

\img{images/r2014}{accel}{Affichage de la résultante des forces de rappel pour quelques sommets aléatoires.}

\img{images/r1822}{uni_potsdam_de}{Ensemble des clés du nom de domaine \unipotsdamde~(Universität Potsdam), qui sont presque toutes au même endroit sur la représentation graphique.}

\img{images/r1822}{sony}{Ensemble des clés de noms de domaines associés à l'entreprise Sony, qui sont toutes au même endroit sur la représentation graphique.}

\img{images/r1822}{byu_edu}{Ensemble des clés du nom de domaine \byuedu~(Brigham Young University), qui sont presque toutes au même endroit sur la représentation graphique.}

\img{images/r1822}{byu_edu_rand}{Ensemble de clés aléatoires aussi nombreuses que les clés du domaine \byuedu~(à comparer avec la figure \ref{img_byu_edu}).}

\img{images/r2014}{ethereal_convent}{Un ensemble de clés assez originales dans le réseau de confiance, qui a été repéré graphiquement avec le logiciel. Elles semblent correspondre aux grades d'une institution religieuse probablement fictive.}


\end{document}

