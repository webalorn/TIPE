\documentclass[a4paper]{article}

\usepackage[utf8]{inputenc}

\usepackage[francais]{babel}

\newcommand{\vtitle}[0]{Représentation et analyse du réseau de confiance OpenPGP}

\usepackage{anysize}

\marginsize{2cm}{2cm}{1cm}{2cm}


\usepackage{url, array, listings, graphicx, framed, relsize}

\sloppy


\title{\vtitle}

\author{Antoine Amarilli}

\date{}


\usepackage{fancyhdr}

\fancypagestyle{plain}{

\fancyhf{}

\fancyhead[L]{\vtitle}

\fancyfoot[L]{Antoine Amarilli}

\fancyfoot[R]{\thepage}

\renewcommand{\headrulewidth}{0.4pt}

\renewcommand{\footrulewidth}{0.4pt}

}

\pagestyle{plain}


\newcommand{\deft}[1]{\textbf{#1}}

\newcommand{\theo}[1]{\emph{#1}}

\newcommand{\code}[2]{
\pagebreak
\begin{framed} 
\lstset{language=C, caption=#1 : #2, label=#1, numbers=left, breakatwhitespace=true, showstringspaces=false, breaklines=true, basicstyle=\scriptsize, }
\lstinputlisting{#1}
\end{framed}
}

\newcommand{\sch}[3]{
\begin{figure}
\centering
\includegraphics{#1_#2.pdf}
\caption{#3}
\label{sch_#2}
\end{figure}
}

\newcommand{\img}[3]{
\begin{figure}
\centering
\includegraphics[angle=90,scale=0.33]{#1_#2.pdf}
\caption{#3}
\label{img_#2}
\end{figure}
}

\newcommand{\tbl}[2]{
\begin{table}[p]
\centering
\input{#1.tex}
\caption{#2}
\label{#1}
\end{table}
}


\urldef\byuedu\url{byu.edu}
\urldef\unipotsdamde\url{uni-potsdam.de}


\begin{document}

\maketitle

\tableofcontents

\pagebreak

\section{Position du problème}

\subsection{Cryptographie asymétrique}

La \deft{cryptographie} s'attache à la protection de la confidentialité, de l'intégrité et de l'authenticité des messages. La \deft{cryptographie asymétrique} procède en associant à chaque utilisateur une \deft{clé privée} gardée secrète et une \deft{clé publique} diffusée à ses correspondants. La clé publique permet de chiffrer des messages et de vérifier des signatures ; la clé privée permet de déchiffrer des messages et d'apposer des signatures (voir figure \ref{sch_alice_bob} p.~\pageref{sch_alice_bob}).

Cependant, une attaque de l'homme du milieu peut être menée lors de l'échange des clés publiques (voir figure \ref{sch_alice_bob_mallory} p.~\pageref{sch_alice_bob_mallory}). Cela rend nécessaire le recours à un canal sans risque d'attaque active.

\subsection{Réseau de confiance}

Plusieurs solutions permettent de pallier ce problème : avoir recours à une autorité de certification centrale, ou utiliser le \deft{réseau de confiance}, que nous étudierons ici. Son fonctionnement est schématisé par la figure \ref{sch_alice_carol_bob} p.~\pageref{sch_alice_carol_bob}. Remarquons qu'il n'offre pas de transitivité (voir figure \ref{sch_alice_carol_dave_zoe} p.~\pageref{sch_alice_carol_dave_zoe}).

En pratique, les correspondants du cryptosystème conservent, en plus de leur clé publique, toutes les signatures apposées par des tiers, et échangent leurs clés par l'intermédiaire de serveurs de clés se synchronisant les uns aux autres.

Lorsque deux personnes désirent signer leurs clés, elles se rencontrent physiquement, offrent chacune une garantie de leur identité (en général une carte d'identité ou un passeport), échangent les empreintes de leurs clés par une fonction de hachage cryptographique pour s'assurer que les clés n'ont pas été falsifiées pendant le transfert, et vérifient que les adresses de courriel sont correctes. Certaines associations organisent des fêtes de signature de clé, où de nombreux participants se rencontrent simultanément pour signer leurs clés.

L'ensemble des clés et des signatures de clés forme le réseau de confiance. Il s'interprète naturellement comme un graphe orienté ayant pour sommets les clés de chiffrement et comme arêtes les signatures d'une clé par une autre.

\subsection{OpenPGP}

\deft{OpenPGP} est un standard de cryptographie asymétrique couramment utilisé actuellement, décrit par \cite{RFC4880}. Les logiciels Pretty Good Privacy (PGP) et GNU Privacy Guard (GPG) en sont des implémentations compatibles, la seconde étant libre et gratuite.

Le site \cite{W1} propose une version téléchargeable et régulièrement mise à jour à partir des serveurs de clés\footnote{La synchronisation n'est cependant pas parfaite ; le serveur à partir duquel \cite{W1} extrait les informations nécessaires n'est pas systématiquement à jour.} de la plus grande composante fortement connexe du réseau de confiance : il fournit la liste des signatures entre clés, ainsi que le nom et l'adresse de courriel indiqués par le créateur de chaque clé.

À l'heure actuelle, le plus grand ensemble fortement connexe comporte un peu plus de $40000$ clés, chacune étant signée en moyenne par environ $10$ autres clés (soit environ $400000$ signatures).

\section{Étude du réseau de confiance}

\subsection{Objectifs du TIPE}

Mon TIPE vise à concevoir un programme permettant de mener des analyses sur le réseau de confiance OpenPGP et d'en fournir une représentation exploitable. En particulier, j'ai cherché à mettre en relation la position d'une clé dans le réseau de confiance et les informations géographiques que permet de déterminer le TLD (\emph{Top Level Domain}) de l'adresse de courriel qui lui est associée.

Une telle étude présente plusieurs applications potentielles. Une corrélation forte entre proximité géographique des clés et proximité dans le graphe permettrait d'inférer avec précision l'origine géographique d'une clé quelconque. Enfin, comme les signatures de clés nécessitent généralement une rencontre physique de leurs propriétaires, une identification de traits caractéristiques du réseau de confiance provenant de cette réalité humaine pourrait permettre de le distinguer de réseaux générés artificiellement : cela offrirait une protection contre un attaquant qui réaliserait un double factice des clés du réseau de confiance, et rendrait possible la détection d'éventuels sous-réseaux suspects dans le graphe.

\subsection{État actuel de la recherche}

Diverses études informelles du réseau de confiance OpenPGP ont été déjà été menées.

Le site \cite{W1} analyse un motif en forme de feuille dans des représentations du réseau de confiance où les clés sont classées par leur distance moyenne aux autres clés, placées sur deux axes orthogonaux, et où un point blanc est positionné à l'intersection des lignes et des colonnes correspondant à des clés qui se sont signées. Différentes variations de cette représentation y sont étudiées (restriction à un TLD, classement par TLD, chemins de longueur 2, autres critères de tri), avec application à un réseau de confiance aléatoire. Le site propose également le logiciel wotsap, qui permet de calculer des statistiques générales sur le réseau de confiance, des statistiques pour une clé, des chemins entre couples de clés, des représentations graphiques, et une liste des signatures qui seraient les plus utiles au réseau de confiance. Enfin, il fournit des exports réguliers du réseau de confiance, que j'ai utilisés.

Le site \cite{W2} propose divers outils liés au réseau de confiance : recherche de chemins entre deux clés, classement des clés par distance moyenne aux autres clés, évolution des statistiques pour les clés individuelles, évolution temporelle du nombre de clés, du nombre de signatures par clés et de la distance moyenne entre clés, distribution du degré des sommets et des distances, comportement du graphe lors de la suppression de sommets aléatoires. Certaines de ces statistiques sont générées avec wotsap, mais les données sont extraites d'autres sources. 

Des analyses plus anciennes sont proposées par \cite{W3}, \cite{W4} et \cite{W5}.

Pour ce qui est du dessin du réseau de confiance, le logiciel sig2dot \cite{W6} permet de convertir des trousseaux de clés OpenPGP en des graphes qui peuvent être fournis à des logiciels généralistes de dessin de graphe. Cependant, le format standard des trousseaux de clés se prête mal à l'importation de l'intégralité du réseau de confiance, et peu de logiciels sont en mesure de représenter un graphe aussi grand en un temps raisonnable.

\section{Analyse du graphe}

\subsection{Importation}

Le graphe du réseau de confiance est importé à partir des données fournies par \cite{W1} sous forme d'un fichier Wotsap (voir \cite{W1b} pour la spécification). Le fichier représentant le graphe de confiance actuel fait environ $1,5$ mégaoctets (le format Wotsap vise à être aussi compact que possible).

Le langage choisi pour la rédaction du programme est C, en raison de la vitesse d'exécution que cela permet d'atteindre, ce qui est nécessaire au vu de la taille du graphe. Le code source complet du programme rédigé pour le TIPE est donné en annexe.

Le graphe est représenté sous la forme de listes d'adjacence indiquant, pour chaque sommet, la liste des arêtes qui en partent et qui y arrivent. Ce choix est motivé par la faible densité du graphe, qui rend cette représentation préférable aux matrices d'adjacence (voir \cite{CFI}, p. 503).

\subsection{Parcours}

J'ai implémenté l'algorithme de parcours du graphe en largeur d'abord, tel que décrit par \cite{IA}. Il permet de calculer la distance des clés du graphe à une clé arbitraire, et donc, en faisant cela pour toutes les clés, la distance moyenne entre les couples de clés. Ce résultat est déjà calculé par wotsap.

\subsection{Lien entre TLD et distances}

Puisque la signature de clés nécessite une rencontre physique entre signataires, que de telles rencontres ont le plus souvent lieu entre habitants d'un même pays, et que le TLD des adresses de courriel correspond parfois à un pays, on peut s'attendre à ce que la distance moyenne entre deux clés d'un même TLD soit plus faible que celle entre deux clés aléatoires (si ce TLD correspond à un pays).

Pour vérifier cette conjecture, le programme calcule la somme des distances entre tous les couples de clés d'un même TLD, et fait la même chose pour les couples de clés d'un sous-ensemble aléatoire du réseau de confiance avec le même nombre de clés. Les distances sont calculées en utilisant l'ensemble du graphe, et non en se restreignant aux arêtes appartenant aux sous-graphes considérés. Les résultats expérimentaux semblent en adéquation avec la conjecture (voir tableau \ref{all_tlds_vs_rand} p.~\pageref{all_tlds_vs_rand}) : la distance moyenne entre clés d'un même TLD est en général plus basse que celle entre clés aléatoires lorsque le TLD correspond à un pays.

On peut aussi penser que la distance entre les clés de deux TLD correspondant à des pays géographiquement proches devrait être plus basse que celle entre des TLD correspondant à des pays géographiquement éloignés. Cependant, cette conjecture n'est pas validée par les résultats expérimentaux (voir tableau \ref{all_tlds_to_all_tlds.custom} p.~\pageref{all_tlds_to_all_tlds.custom}). Une explication de ce phénomène est la difficulté que représente la comparaison des distances entre deux couples de TLD, puisque la structure individuelle de chaque TLD influe sur les résultats obtenus.

\section{Représentation du graphe}

\subsection{Critères esthétiques}

La représentation d'un graphe peut se faire suivant différents critères. Parmi les plus courants, citons (voir \cite{GD}, p. 12-16) :

\begin{itemize}
\item Minimisation du nombre de croisements entre les arêtes (une solution sans croisement n'est possible que pour les graphes planaires).
\item Respect d'une contrainte sur le rapport hauteur/largeur du dessin.
\item Représentation des arêtes par des segments ayant autant que possible la même longueur.
\item Dessin des arêtes avec des lignes aussi droites que possible.
\item Maximisation de l'angle entre les représentations de deux arêtes incidentes à un même sommet.
\item Respect des symétries.
\end{itemize}

Différentes approches générales peuvent être retenues pour le dessin. Par exemple, on peut décider de représenter les arêtes par des segments quelconques, ou par des successions de segments verticaux ou horizontaux\footnote{De tels dessins sont utiles pour l'intégration à très grande échelle (VLSI), selon \cite{IGT}, p.~199.}.

Le critère esthétique retenu pour le dessin du réseau de confiance est la minimalité des variations entre la longueur des arêtes. En effet, l'objectif principal est l'étude des distances entre sommets, d'où la volonté de lier les distances sur la représentation aux distances dans le graphe. Les croisements n'ont que peu d'importance car les arêtes sont trop nombreuses pour être toutes représentées d'une manière lisible.

\subsection{Choix d'un algorithme}

Les algorithmes de dessin de graphe sont nombreux. Les différences entre eux concernent principalement les critères esthétiques qu'ils permettent de respecter, les types de graphe auxquels ils s'appliquent, et leurs performances. Un résumé est proposé par \cite{GD}, p. 38.

L'algorithme force-directed a été retenu pour plusieurs raisons. Tout d'abord, il peut être appliqué à des graphes quelconques, au contraire d'autres algorithmes nécessitant des propriétés particulières que le graphe du réseau de confiance ne présente pas (caractère planaire, acyclique, etc.). Il suit également le critère esthétique choisi. Enfin, sa simplicité le rend assez performant en pratique.

\subsection{Algorithme force-directed}

L'algorithme force-directed modélise le graphe étudié comme un système physique, en considérant les arêtes comme des ressorts de longueur à vide fixée attachés à des masses représentant les sommets. À chaque itération, le système calcule la résultante des forces exercées sur chaque sommet (voir figure \ref{img_accel} p.~\pageref{img_accel}) et le déplace légèrement dans la direction de la résultante. L'énergie potentielle des ressorts diminue au cours du temps, jusqu'à atteindre un minimum local qui est une position d'équilibre du système physique (et un dessin esthétiquement plaisant du graphe).

On ajoute habituellement une force de répulsion électrostatique entre les sommets pour éviter qu'ils ne s'entassent au centre. Cependant, le graphe du réseau de confiance est peu dense (le nombre total d'arêtes est petit devant le carré du nombre de sommets), donc le temps nécessaire au calcul de la force de répulsion, qui s'exerce entre tout couple de sommets, serait très grand devant celui nécessaire au calcul des forces de rappel des ressorts qui s'exercent pour chaque arête. Aussi, pour que les performances restent acceptables, le programme se limite au calcul des forces de rappel, ce qui permet d'avoir plusieurs itérations par seconde au lieu d'une itération au bout de quelques minutes.

Quelques adaptations ont dû être faites pour obtenir malgré tout une représentation exploitable. Afin de limiter la tendance à l'agglutinement au centre, les clés sont initialement disposées sur un cercle grand devant la longueur à vide des ressorts. Les clés se déplacent suivant la résultante des forces non pas d'un petit déplacement fixe, mais d'un déplacement aussi grand que nécessaire tant que cela contribue à la réduction de l'énergie potentielle ; cela semble empiriquement favoriser l'apparition d'alignements de clés en périphérie du graphe. Au contraire, certaines optimisations qui faisaient diminuer l'énergie potentielle plus vite ont dû être abandonnées car elles rendaient le graphe illisible...

\subsection{Interface}

Les bibliothèques SDL, SDL\_ttf, SDL\_Input et SDL\_Input\_TTF sont utilisées pour représenter le graphe au fur et à mesure de l'exécution de l'algorithme.

L'interface développée, outre l'affichage du graphe, offre de nombreuses commandes récapitulées dans le tableau \ref{commands} (p.~\pageref{commands}).

\subsection{Observations}

Le phénomène de proximité entre clés d'un même TLD national dans le graphe s'observe aussi sur la représentation graphique après exécution de l'algorithme, comme le montrent les tableaux \ref{dist_vs_gdist_before} p.~\pageref{dist_vs_gdist_before} (avant l'exécution de l'algorithme) et \ref{dist_vs_gdist_after} p.~\pageref{dist_vs_gdist_after} (après l'exécution).

Lors de l'évolution du dessin, on observe que certaines clés mal intégrées restent en périphérie de la représentation graphique. Il s'agit en général de clés reliées au reste du réseau par un seul maillon. Ces clés sont le plus souvent membres du même TLD, voire du même domaine. Dans certains cas, toutes les clés d'un même domaine se retrouvent au même endroit sur le dessin (voir figures \ref{img_uni_potsdam_de} p.~\pageref{img_uni_potsdam_de}, \ref{img_sony} p.~\pageref{img_sony}, \ref{img_byu_edu} p.~\pageref{img_byu_edu}, \ref{img_byu_edu_rand} p.~\pageref{img_byu_edu_rand} et \ref{img_ethereal_convent} p.~\pageref{img_ethereal_convent}).

De manière générale, il y a une différence graphique observable à l'œil nu entre les ensembles de clés correspondant à un TLD national et les ensembles de clés sélectionnés aléatoirement ; les ensembles correspondant à un TLD national comprennent le plus souvent la totalité ou la quasi-totalité de plusieurs ensembles de clés en périphérie. On peut par exemple comparer la figure \ref{img_at} (p.~\pageref{img_at}), qui met en évidence les clés autrichiennes, à la figure \ref{img_rnd_at} (p.~\pageref{img_rnd_at}), qui met en évidence le même nombre de clés aléatoires.

Même vers le centre de la représentation, où des clés se retrouvent graphiquement proches bien qu'éloignées dans le graphe, on observe que la répartition des différents TLD n'est pas vraiment homogène, comme l'illustre la figure \ref{img_all_colors} p.~\pageref{img_all_colors}.

Le logiciel permet aussi de repérer quelques curiosités du réseau de confiance. Voir par exemple l'image \ref{img_ethereal_convent} p.~\pageref{img_ethereal_convent}.

\section{Prolongements envisageables}

Des améliorations de différents types pourraient être apportées au programme. Certains choix d'implémentation se sont révélés peu judicieux ; un bon nombre de fonctions pourrait être regroupé en fonctions génériques ; il faudrait à plusieurs endroits supprimer les limites stockées dans des constantes globales et utiliser malloc.

Pour l'ajout d'une force de répulsion électrostatique, l'utilisation de structures de données telles que des quadtrees pourrait permettre de regrouper les clés selon leur position sur la représentation graphique. Ainsi, les effets de la répulsion pourraient être approximés en représentant les ensembles de clés éloignées de la clé d'étude par des masses ponctuelles pour réduire les calculs.

Une telle adaptation de l'algorithme force-directed est mise en œuvre par le projet FADE \cite{FADE}, qui affirme atteindre des temps d'exécution en $\Theta(n \log n)$.

\pagebreak

\begin{thebibliography}{WW}
	\bibitem {GD} Giuseppe Di Battista, Peter Eades, Roberto Tamassia, Ioannis G. Tollis, \emph{Graph Drawing: Algorithms for the Visualization of Graphs}. Prentice Hall, Upper Saddle River, 1999.
        \bibitem {IGT} Gary Chartrand, \emph{Introductory Graph Theory}. Dover, New York, 1985.
        \bibitem {TC} Jean-Guillaume Dumas, Jean-Louis Roch, Éric Tannier, Sébastien Varrette, \emph{Théorie des codes : Compression, cryptage, correction}. Dunod, Paris, 2007.
        \bibitem {CFI} Alfred Aho, Jeffrey Ullman, \emph{Concepts fondamentaux de l'informatique}. Trad. X. Cazin, I. Gourhant, J.-P. Le Narzul. Dunod, Paris, 1993.
        \bibitem {IA} Thomas H. Cormen, Charles E. Leiserson, Ronald L. Rivest, Clifford Stein, \emph{Introduction to Algorithms}, deuxième édition. MIT Press et McGraw-Hill, Cambridge, Massachusetts, 2001.
        \bibitem {W1} Wotsap [En ligne]. Jörgen Cederlöf, 2006. Disponible à l'adresse : \url{http://www.lysator.liu.se/~jc/wotsap/} 
        \bibitem {W2} PGP pathfinder and key statistics [En ligne]. Henk P. Penning, 2009. Disponible à l'adresse : \url{http://pgp.cs.uu.nl/}
        \bibitem {W3} The Footsie Web of Trust analysis [En ligne]. Matthew Wilcox, 2009. Disponible à l'adresse : \url{http://www.parisc-linux.org/~willy/wot/footsie/}
        \bibitem {W4} Keyanalyse [En ligne]. M. Drew Streib, 2002. Disponible à l'adresse : \url{http://dtype.org/keyanalyze/}
        \bibitem {W5} PGP Web of Trust Statistics [En ligne]. Neal McBurnett, 1997. Disponible à l'adresse : \url{http://bcn.boulder.co.us/~neal/pgpstat/} 
        \bibitem {W6} Sig2dot GPG/PGP Keyring Graph Generator [En ligne]. Nathaniel E. Barwell, 2002. Disponible à l'adresse : \url{http://www.chaosreigns.com/code/sig2dot/} 
        \bibitem {FADE} FADE [En ligne]. Aaron J. Quigley, 2006. Disponible à l'adresse : \url{http://www.csi.ucd.ie/staff/aquigley/home/?Research:Projects:FADE}
        \bibitem {RFC4880} RFC 4880 [En ligne]. J. Callas, L. Donnerhacke, H. Finney, D. Shaw, R. Thayer, 2007. Disponible à l'adresse : \url{http://tools.ietf.org/html/rfc4880}
        \bibitem {W1b} The Web of Trust .wot file format, version 0.2 [En ligne]. Jörgen Cederlöf, 2004. Disponible à l'adresse : \url{http://www.lysator.liu.se/~jc/wotsap/wotfileformat.txt}
\end{thebibliography}

% \listoffigures

% \listoftables


\sch{schemas/sch}{alice_bob}{Schéma de principe de la communication à l'aide de la cryptographie asymétrique. Les clés A1 et A2 sont respectivement les clés privée et publique d'Alice, B1 et B2 celles de Bob. Les flèches en pointillés indiquent le transfert de données sur un canal vulnérable aux attaques passives. Le protocole assure à Alice que son message n'est lisible que par Bob, et assure à Bob l'authenticité du message reçu.}

\sch{schemas/sch}{alice_bob_mallory}{Schéma de principe de l'attaque de l'homme du milieu, menée par Mallory. Les clés A1 et A2 sont respectivement les clés privée et publique d'Alice, B1 et B2 celles de Bob, et A'1, A'2, B'1, B'2 des clés factices créées par Mallory. Les flèches en pointillés indiquent le transfert de données sur un canal vulnérable aux attaques actives. En se faisant passer pour Bob auprès d'Alice et pour Alice auprès de Bob, Mallory est en mesure de lire et de modifier le message.}

\sch{schemas/sch}{alice_carol_bob}{Schéma de principe du réseau de confiance. Les flèches bleues, vertes et rouges indiquent la signature de clé, la confiance en une personne, et l'assurance de la validité de la clé respectivement. Alice a vérifié la clé de Carol (elle a donc signé la clé de Carol) et a confiance en Carol, et Carol a vérifié la clé de Bob (elle a donc signé la clé de Bob), donc Alice a une garantie de la validité de la clé de Bob.}

\sch{schemas/sch}{alice_carol_dave_zoe}{Schéma de principe de la non-transitivité du réseau de confiance. La légende est celle de la figure \ref{sch_alice_carol_bob}. Alice a vérifié la clé de Carol et a confiance en elle, et celle-ci a vérifié la clé de Dave. Alice a donc une garantie de la validité de la clé de Dave, mais pas de celle de Zoé puisqu'Alice n'a pas confiance en Dave. Voir \cite{TC}, p. 209 et 312.}


\tbl{all_tlds_vs_rand}{Distance moyenne entre tout couple de clés pour chaque TLD, comparé aux distances pour un sous-ensemble aléatoire de clés de même taille. Les colonnes indiquent respectivement le nombre de clés dans le TLD, la distance moyenne entre tout couple de clés du TLD, la distance moyenne entre tout couple de clés du sous-ensemble aléatoire, et la différence de ces deux colonnes. Pour les TLD correspondant à un pays, la distance moyenne du TLD est en général plus basse que celle de l'ensemble de clés aléatoires.}

\tbl{all_tlds_to_all_tlds.custom}{Distance moyenne entre les clés d'un TLD et celles d'un autre TLD. Aucune tendance notable ne semble pouvoir être observée. Noter que le tableau n'est pas symétrique, car le graphe du réseau de confiance est orienté. Les distances sont indiquées en partant du TLD de la ligne pour aller jusqu'au TLD de la colonne.}

\tbl{dist_vs_gdist_before}{Distance moyenne graphique (euclidienne) entre tout couple de clés pour chaque TLD, comparé aux distances pour un sous-ensemble aléatoire de clés de même taille, avant lancement de l'algorithme force-directed. Les colonnes sont les mêmes que celles du tableau~\ref{all_tlds_vs_rand}, à ceci près qu'il s'agit ici de distances graphiques. Aucune tendance notable ne semble pouvoir être observée.}

\tbl{dist_vs_gdist_after}{Distance moyenne graphique (euclidienne) entre tout couple de clés pour chaque TLD, comparé aux distances pour un sous-ensemble aléatoire de clés de même taille, après exécution de l'algorithme force-directed pendant quelques heures. Les colonnes sont les mêmes que celles du tableau~\ref{dist_vs_gdist_before}. Pour les TLD correspondant à un pays, la distance moyenne du TLD est en général plus basse que celle de l'ensemble de clés aléatoires.}

\begin{table}[p]
\centering
\begin{tabular}{|c|l|}
\hline {\bf Entrée} & {\bf Effet} \\
\hline Clic gauche & Sélection de clé(s) \\
\hline Clic droit & Déplacement de clé(s) \\
\hline Clic central & Déplacement de la vue \\
\hline Molette & Zoom \\
\hline a & Tout sélectionner \\
\hline c & Colorier les clés \\
\hline d & Calculer les distances entre clés \\
\hline e & Sélection par TLD ou adresse de courriel \\
\hline f & Marquage des clés de départ pour les calculs de distance \\
\hline g & Affichage de la résultante des forces \\
\hline i & Affichage des identifiants de clés \\
\hline k & Sélection par identifiant \\
\hline l & Recalcul manuel de la résultante \\
\hline m & Déplacement manuel suivant la résultante \\
\hline n & Affichage des noms et adresses de courriel \\
\hline q & Quitter \\
\hline r & Sélection de clés aléatoires \\
\hline s & Sélection des clés ayant signé les clés sélectionnées \\
\hline t & Affichage du nombre de clés dans la sélection \\
\hline v & Inversion de la sélection \\
\hline x & Activation ou désactivation de l'algorithme force-directed \\
\hline z & Zoom automatique \\
\hline / & Remise à zéro des opérateurs \\
\hline + & Opérateur union \\
\hline - & Opérateur différence ensembliste \\
\hline * & Opérateur intersection \\
\hline \verb?\? & Opérateur différence symétrique \\
\hline A & Tout sélectionner \\
\hline C & Coloriage rapide \\
\hline D & Suppression des marques de départ et d'arrivée \\
\hline F & Marquage des clés d'arrivée pour les calculs de distance \\
\hline G & Masquage de la résultante des forces \\
\hline I & Masquage des identifiants de clés \\
\hline L & Recalcul manuel de la résultante (toutes les clés) \\
\hline M & Déplacement manuel suivant la résultante (toutes les clés) \\
\hline N & Masquage des noms et adresses de courriel \\
\hline S & Sélection des clés ayant été signées par les clés sélectionnées \\
\hline Z & Centrer la vue \\
\hline Ctrl+A & Calcul de données pour les tableaux \ref{all_tlds_vs_rand}, \ref{dist_vs_gdist_before} et \ref{dist_vs_gdist_after} \\
\hline Ctrl+B & Calcul de données pour le tableau \ref{all_tlds_to_all_tlds.custom} \\
\hline Ctrl+C & Arrangement des clés sélectionnées en cercle \\
\hline
\end{tabular}
\caption{Liste des commandes du logiciel.}
\label{commands}
\end{table}


\img{images/r2002}{at}{Position des clés autrichiennes (en noir) dans le réseau de confiance.}

\img{images/r2002}{rnd_at}{Position d'un sous-ensemble de clés aléatoires aussi nombreuses que les clés autrichiennes, à comparer avec la figure \ref{img_at}. On remarque que les clés autrichiennes ont davantage tendance à former des alignements.}

% \img{images/r1822}{ch}{Position des clés suisses (en noir) dans le réseau de confiance. On remarque des alignements de clés.}

% \img{images/r1822}{ch_zoom}{Zoom sur un alignement de la figure \ref{img_ch}.}

% \img{images/r1822}{keys_colors}{Ensemble du réseau de confiance, avec coloriage des clés selon leur TLD. La légende associant couleurs et TLD est indiquée par les clés dont l'adresse de courriel est affichée.}

% \img{images/r1822}{center_colors}{Zoom sur le centre du réseau de confiance (voir figure \ref{img_keys_colors}), avec coloriage des clés selon leur TLD. La répartition des TLD n'est pas la même en tous points.}

\img{images/r2014}{all_colors}{Centre du réseau de confiance, avec coloriage des clés selon leur TLD. On observe que les TLD ne sont pas répartis de façon homogène.}

\img{images/r2014}{accel}{Affichage de la résultante des forces de rappel pour quelques sommets aléatoires.}

\img{images/r1822}{uni_potsdam_de}{Ensemble des clés du nom de domaine \unipotsdamde~(Universität Potsdam), qui sont presque toutes au même endroit sur la représentation graphique.}

\img{images/r1822}{sony}{Ensemble des clés de noms de domaines associés à l'entreprise Sony, qui sont toutes au même endroit sur la représentation graphique.}

\img{images/r1822}{byu_edu}{Ensemble des clés du nom de domaine \byuedu~(Brigham Young University), qui sont presque toutes au même endroit sur la représentation graphique.}

\img{images/r1822}{byu_edu_rand}{Ensemble de clés aléatoires aussi nombreuses que les clés du domaine \byuedu~(à comparer avec la figure \ref{img_byu_edu}).}

\img{images/r2014}{ethereal_convent}{Un ensemble de clés assez originales dans le réseau de confiance, qui a été repéré graphiquement avec le logiciel. Elles semblent correspondre aux grades d'une institution religieuse probablement fictive.}



\code{main.c}{Fonction main}

\code{main.h}{Variables et constantes globales}

\code{load.c}{Fonctions d'initialisation}

\code{structures.c}{Structures fondamentales}

\code{marks.c}{Marquage de clés}

\code{distances.c}{Calculs de distances}
  
\code{force.c}{Algorithme force-directed}

\code{events.c}{Gestion des événements}

\code{graphics.c}{Affichage graphique}

\code{misc.c}{Fonctions diverses}

\end{document}

