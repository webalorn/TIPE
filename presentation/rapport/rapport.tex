% !TeX program = xelatex
% !TeX encoding = UTF-8
% !TeX spellcheck = fr_FR

\documentclass[a4paper,twoside]{article} % removed : french
\usepackage{polyglossia}
\setdefaultlanguage{french}
\usepackage[T1]{fontenc}
\usepackage[utf8]{inputenc} %encodage utf8
\usepackage{fontspec}
\defaultfontfeatures{Ligatures={Common,TeX}}
\usepackage[pdfusetitle,hidelinks]{hyperref}
\usepackage{color}
\usepackage{graphicx}
\usepackage[left=2cm,right=2cm,top=2cm,bottom=2cm]{geometry}
\usepackage{tikz}
\usepackage{mathdots}
\usepackage{yhmath}
\usepackage{cancel}
\usepackage{color}
\usepackage{siunitx}
\usepackage{array,booktabs}
\usepackage{multirow}
\usepackage{amssymb}
\usepackage{gensymb}
\usepackage{tabularx}
\usetikzlibrary{fadings}

\usepackage{verbatim}
\usepackage{scalefnt}
\usepackage{fp}

\usepackage[procnames]{listings}
\usepackage{amssymb}
\usepackage{hhline,colortbl}
\usepackage{algpseudocode}

\usepackage{tcolorbox}  %pour créer des cadres, très riche, regarder sur le net pour plus d'exemples
\tcbuselibrary{breakable,listings} %permet l'insertion de programme dans des boites 
\tcbuselibrary{theorems}

\graphicspath{{./figures/}}
\tikzset{every picture/.style={line width=0.75pt}} %set default line width to 0.75pt 

\title{Reconnaissance efficace d’objets sous-marins}
\author{Vallaeys Théophane}
\date{2019-2020}

\begin{document}
	\maketitle
	
	Les milieux sous-marins sont des environnements complexes pouvant contenir de nombreuses espèces animales et végétales, objets et débris. Être capable de détecter et identifier correctement de tels objets sur des images sous-marines possède des applications directe pour l'automatisation de drones, ou encore l'analyse non-destructive des population.
	
	Je me suis ainsi intéressé à la conception d'un système capable de classifier des images parmi un \textbf{nombre de classes important}, sous la contrainte d'être capable d'\textbf{apprendre à identifier de nouveaux objets rapidement}, sans un lourd ré-entrainement nécessité par des réseaux de neurones artificiels classiques. L'intérêt est de le rendre adaptatif à la reconnaissance de nouvelles espèces marines ou d'objets précis.
	
	Je supposerai par la suite que le lecteur à possède des connaissances sur les bases de l'apprentissage automatique, et plus particulièrement sur le fonctionnement des réseaux de neurones artificiels.
	
	\tableofcontents
	
	\section{Méthodes simples de classification}
	
	
	
	\section{Transformation de l'image à l'aide d'un réseau siamois}
	
	\section{Classification rapide à l'aides des plus proches voisins}
	
	
	
\end{document}